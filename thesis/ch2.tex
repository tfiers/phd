
\section{Neuron model}

We choose to simulate the `AdEx' neuron model, or the `adaptive exponential integrate-and-fire' neuron.
This is a leaky-integrate-and-fire (LIF) neuron model, with two additions.
First, the full upstroke of each spike is simulated, as an exponential runoff.
Second, an extra dynamic variable is added: the adaptation current.
This allows the simulation of many non-linear effects of real neurons, like spike-rate adaptation and post-inhibitory rebound. (Note however that we do not focus on any of these effects in this thesis).

The AdEx model consists of two differential equations, one to simulate the membrane voltage $V$, and one for the adaptation current $w$:

\begin{align*}
    \d{V} &=  \\
    \d{w} &=
\end{align*}

We solve these equations using first-order (Euler) integration.

In addition to the two differential equations, the AdEx model also consists of an instantaneous reset condition. When the membrane voltage $V$ reaches a certain threshold, a spike is recorded, $V$ is reset, and $w$ is increased:

\begin{align*}
    \text{if}\ V > \theta\ \text{then:}\
    & V \leftarrow V_r \\
    & w \leftarrow w + \Delta w
\end{align*}


\section{Input spikes}

In our simplest experimental setup, we simulate just one AdEx neuron.
Its input is provided by an array of $N$ Poisson neurons, i.e. they each generate spike trains according to a Poisson process. We call this the `N-to-1' setup.

The inter-event intervals of a Poisson process follow an exponential distribution.
We use that fact to generate spike trains: we draw samples from $\mathrm{Exp}(\lambda)$ (with $\lambda$ the desired firing rate), and cumulatively sum up these intervals  to obtain spike times. This is done until we have reached the desired input train duration.


\section{Voltage imaging}

The signals detected by a light microscope in a voltage imaging setup are not the same as the real membrane voltage signals of which they are a reflection.

We model this lossy transformation by simply adding Gaussian noise to our simulated membrane voltage. As in the voltage imaging literature, we quantify the amount of this  noise by a `spike-SNR' measure (spike signal-to-noise ratio). This is defined as the height of an average spike relative to the standard deviation of the noise.

A more realistic model of the voltage-imaging transformation would also incorporate the exponential decay over time of the SNR, and the short-term 'smearing in time' of voltage indicators. The latter could be done by passing the voltage signal through a linear filter with non-instantaneous impulse response.
