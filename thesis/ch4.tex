


\section{STA Template correlation}

\begin{itemize}
    \item Idea for the two-pass test
    \item Template examples. Both ideal and template found with first-pass (STA height-only)
    \item Peformance for different N, \& comparison with previous method
\end{itemize}



\section{Fitting a full STA model}

\begin{itemize}
    \item Model design
    \item Iterative model fitting
    \item Problem of overfitting, and parameter-constraints to solve it
    \item An advantage: fit parameters (like transmission delay and time constants) are biologically ~meaningful
    \item Perfomance for different N, \& comparison with previous methods
\end{itemize}


\section{Linear regression of the upstroke}

\begin{itemize}
    \item Non-STA method: concatenated individual windows as (X, y)
    \item Examples of pooled windows, and fits
    \item Mention the problem of unknown transmission delays
    \item Perfomance for different N, \& comparison with previous methods
\end{itemize}


% \section{Clustering, \& Hierarchical model fitting}

% \begin{itemize}
%     \item (Time-permitting)
% \end{itemize}


% \section{Zhou/Cai's `Spike-triggered regression'}

% \begin{itemize}
%     \item (Time-permitting)
% \end{itemize}


\section{Computational cost}

\begin{itemize}
    \item Timings of each method, extrapolation for larger number of tested connections
\end{itemize}

\section{Summary}

\begin{itemize}
    \item Conclusions of the N-to-1 experiment
    \item Leadup to the network experiments: what we could not yet test (the problem of indirect connections, as e.g. identified in the connectomics challenge)
\end{itemize}
