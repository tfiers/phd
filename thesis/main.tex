\documentclass[a4paper, oneside, 11pt]{memoir}
\input{totex/Settings.tex}
\input{totex/Commands.tex}


\usepackage{tikz}
\usepackage{pgfplots}
\pgfplotsset{compat=1.3}

\definecolor{LighterBlack}{gray}{0.4}


% No '(2.8)' for equations, rather just '(10)'
\counterwithout{equation}{chapter}
\counterwithout{figure}{chapter}
\counterwithout{table}{chapter}
% for listings, this is done by lstset numberbychapter=false, in totex/Settings.tex

\renewcommand{\d}[2][t]{\ensuremath{\frac{\mathrm{d}#2}{\mathrm{d}#1}}}
\newcommand*{\dd}[1]{\ensuremath{\frac{\mathrm{d}}{\mathrm{d}#1}}}

\newcommand*{\nburl}[1]{{\footnotesize\texttt{\href{https://tfiers.github.io/phd/nb/#1.html}{\replunderscores{#1}}}}}

% Thx https://stackoverflow.com/a/57028315/2611913
% (with addition of \allowbreak)
\makeatletter
\newcommand{\replunderscores}[1]{\expandafter\@repl@underscores#1_\relax}
\def\@repl@underscores#1_#2\relax{%
    \ifx \relax #2\relax
        #1%
    \else
        #1%
        \textunderscore\allowbreak
        \@repl@underscores#2\relax
    \fi
}
\makeatother

% \newcommand{\mpl}[1]{#1}
% \newcommand{\mpl}[1]{\IfStrEqCase{b}{{b}{BB}{a}{AA}}}
%         {blue}{blueyyy}%
% %         {blue}{\textcolor[HTML]{1F77B4}{blue}}%
% %         {orange}{\textcolor[HTML]{FF7F0E}{orange}}%
% %         {green}{\textcolor[HTML]{2CA02C}{green}}%
% %         {red}{\textcolor[HTML]{D62728}{red}}%
%     }[#1]%
% }

% `\IfStrEqCase' from xstring didn't work (it did in an MWE; but some bad interaction)
%  ("Illegal parameter number in definition of \@tempa")
% (https://tex.stackexchange.com/a/61602/153868).
%
% New style, using xparse and latex3 cmds,
% from e.g. https://tex.stackexchange.com/a/61603/153868, does work
\ExplSyntaxOn
\NewDocumentCommand{\mpl}{m}
  {
   \str_case:nnF { #1 }
     {
      {blue} {\textcolor[rgb]{0.106, 0.408, 0.616}{blue}}
      {orange} {\textcolor[rgb]{0.871, 0.431, 0.047}{orange}}
      {green} {\textcolor[rgb]{0.149, 0.545, 0.149}{green}}
      {red} {\textcolor[rgb]{0.729, 0.133, 0.137}{red}}
      {purple} {\textcolor[rgb]{0.506, 0.353, 0.643}{purple}}
      {brown} {\textcolor[rgb]{0.478, 0.294, 0.255}{brown}}
     }
     {#1}
  }
\ExplSyntaxOff
% mpl colours, but darkened a bit (to make thin text visually match thick lines).
% Calculated at 2023-08-30__STA_examples.

% Command that does nothing.
% To be able to have autocomplete of sidecaption labels for vscode LaTeX-Workshop
% [https://github.com/James-Yu/LaTeX-Workshop/wiki/Intellisense#references]
\newcommand*{\linelabel}[1]{}
% (using a self-defined command ("\sclabel") didn't work. So using this one, that had
% not yet a def here).

% For "1st" etc. (May not use `^{}')
\newcommand{\ts}{\textsuperscript}

\newcommand*{\exc}{\text{exc}}
\newcommand*{\inh}{\text{inh}}
\newcommand*{\syn}{\text{syn}}

\newcommand*{\maxF}{\ensuremath{\max{}\,F_1}\xspace}

\newunicodechar{≤}{\ensuremath{\le}}
\newunicodechar{≥}{\ensuremath{\ge}}
\newunicodechar{≠}{\ensuremath{\neq}}
\newunicodechar{≈}{\ensuremath{\approx}}
\newunicodechar{≡}{\ensuremath{\equiv}}
\newunicodechar{Δ}{\ensuremath{\Delta}}
% Weird: Typing `\Delta' in plain text works, but
%   \newunicodechar{Δ}{\Delta}
% doesn't.
\newunicodechar{α}{\ensuremath{\alpha}}
\newunicodechar{β}{\ensuremath{\beta}}
\newunicodechar{θ}{\ensuremath{\theta}}
\newunicodechar{μ}{\ensuremath{\mu}}
\newunicodechar{σ}{\ensuremath{\sigma}}
\newunicodechar{τ}{\ensuremath{\tau}}
\newunicodechar{δ}{\ensuremath{\delta}}
\newunicodechar{ε}{\ensuremath{\varepsilon}}
\newunicodechar{λ}{\ensuremath{\lambda}}
% \newunicodechar{×}{\ensuremath{\times}}  % not needed (outside math)
\newunicodechar{×}{\ensuremath{\times}}
\newunicodechar{₁}{\ensuremath{_1}}
\newunicodechar{₂}{\ensuremath{_2}}
\newunicodechar{ₘ}{\ensuremath{_\text{m}}}
\newunicodechar{ₛ}{\ensuremath{_\text{s}}}
\newunicodechar{ᵢ}{\ensuremath{_{i}}}
\newunicodechar{ⱼ}{\ensuremath{_{j}}}
\newunicodechar{–}{\ensuremath{-}}
\newunicodechar{·}{\ensuremath{\cdot}}
\newunicodechar{→}{\ensuremath{\rightarrow}}
\newunicodechar{←}{\ensuremath{\leftarrow}}
\newunicodechar{∀}{\ensuremath{\forall}}
\newunicodechar{∈}{\ensuremath{\in}}
\newunicodechar{∞}{\ensuremath{\infty}}
\newunicodechar{…}{\ensuremath{\mathellipsis}}


\addbibresource{references.bib}

\graphicspath{{figs/}}

\begin{document}

\chapter*{Preamble}

Figures and tables in this document are often accompanied by a link to the Jupyter Notebook that generated them. These notebooks contain more information than can be conveyed in just the caption. These links look like, e.g:
\nburl{2021-12-06__local_HH_dV_shape}.
To visit these links, in the digital version of this document, just click them. In the paper version, convert them to an URL as follows: {\small \url{https://tfiers.github.io/phd/nb/2021-12-06__local_HH_dV_shape.html}}.


\chapter{Introduction}

\begin{itemize}
    \item Problem statement
    \item Opportunity of voltage imaging, core idea of inverting the causal `presynaptic spike → postsynaptic PSP' relation
    \item Overview of thesis
\end{itemize}


% \begin{itemize}
%     \item Problem statement
%     \item Opportunity of voltage imaging, core idea of inverting the causal `presynaptic spike → postsynaptic PSP' relation
%     \item Overview of thesis
% \end{itemize}

% \section{Connectomics}

% \begin{itemize}
%     \item Motivation
%     \item Ground-truth connectomics: tracing of electron microscopy and fluoresent injection imaging
%     \item Necessity of connection \emph{inference}
%     \item Mention `invasive' connection testing (stimulate one cell, record possible neighbours)
%     \item Limitations of `connectomics', and of inferred vs `actual' connectomics.\\
%         Terminology, e.g. `functional connectomics'
% \end{itemize}


% \section{Voltage imaging}

% \begin{itemize}
%     \item Technologies (from dyes to GEVIs)
%     \item Specs: cell yield, tissue depth, recording duration, SNR, species
%     \item ..and growth of these over time, and comparison with calcium imaging.\\
%         To extrapolate how these might advance in the future
%     \item Comparison with other recording techniques: ephys, calcium imaging, (and briefly mention coarser methods)
% \end{itemize}


% \section{Network inference}

% \begin{itemize}
%     \item Working with events/spikes only, versus working with continuous signals; or a hybrid as here.
%     \item Overview of the spikes-only methods
%     \item The connectomics competition, and the findings about the best methods
%     \item Mention other application domains, like gene regulatory networks
%     \item Conclusion: connection inference by spikes is poor (Ila Fiete etc), and Vm imaging offers unique advantage (causality)
%     % \item 'Spike-triggered voltage regression' of Zhou/Cai
% \end{itemize}



The first section here gives an overview of the idea of this thesis, and its background. The second section delves a bit deeper into voltage imaging technology.



\section{Overview}

Systems neuroscience studies the links between (1) an animal’s behaviour, (2) the activity of its neurons, and (3) how these neurons are connected. Currently, only the first two can be observed simultaneously, using \emph{in vivo} recordings of neural activity. Observing the connections between neurons, on the other hand, requires imaging brain slices, and thus killing the animal. In addition, such a wire-tracing process is costly and time-consuming.

In this thesis, we aim to develop algorithms that infer the connections between neurons based on recordings of their voltages, instead of post-mortem imaging. We believe this is possible because the activity of neurons is mainly determined by the connections between them, and because recent advances in recording technology are yielding, for the first time, the necessary quality of data to solve this problem.

Such an algorithm then allows for fast and cheap estimation of the neural wiring in behaving animals, throughout their lifetime and across experiments. This could allow systems neuroscientists – whether they study e.g. memory, addiction, or movement disorders – to find answers to their questions linking brain wiring and brain (dys)function in a manner more straightforward than before.


\section{Recording neural activity}

There are currently two methods in popular use to record the activity of multiple individual neurons, in vivo: calcium imaging and extracellular electrode recordings.\cite{Carter2015GuideResearchTechniques,Scanziani2009ElectrophysiologyAgeLight}
The strength of each method is the weakness of the other. Both are performed through a small, surgically created hole in the skull.

In calcium imaging, neurons are filled with a calcium indicator or “dye” – a molecule that becomes fluorescent when calcium binds to it. When a neuron sends an action potential (a “spike”), its cell body is briefly flooded with calcium. A dyed neuron that has just sent a spike thus becomes fluorescent for a short while. To record activity, laser light is focused in a point and scanned across a volume of brain tissue. Neurons that recently spiked will send light back, which is captured to yield a three-dimensional movie of neural activity. This allows scientists to observe large numbers of neurons – namely all active neurons in the volume. An additional advantage is that calcium indicators can be genetically targeted so that they only occur in specific neuron types of interest, providing a focused view. A major disadvantage however is that multiple spikes fired from a neuron in rapid succession cannot be easily distinguished, as the calcium effects of each spike are slow and combine non-linearly.

Extracellular electrode recordings on the other hand have a much finer time resolution and every spike is individually distinguishable. They work by inserting long, thin implants in the brain, that have many electrodes exposed on their surface. Each electrode measures the local electric field potential, and thereby picks up the spikes from nearby neurons. The increased time resolution comes at the cost of only sampling a small subset of the neurons in the areas of interest, not knowing exactly where those neurons are, and not being able to target neural subtypes specifically.

Calcium imaging thus provides good spatial information but has a low temporal resolution, whereas extracellular electrode recordings present the opposite trade-off: precise spike counts and timings, but limited spatial information and sampling of neurons. Recently, a recording technique is emerging that combines the advantages of both.


\FloatBarrier
\section{Voltage imaging}

\begin{figure}
    \includegraphics[w=1.5]{VI-sigs}
    \captionn{Voltage imaging}{
        Left: schematic of an example voltage indicator molecule embedded in the cell membrane. Middle: the imaging setup and an example neuron image. Right: an example voltage imaging trace and a simultaneous intracellular electrode recording.\\
        Adapted from \cite{Villette2019UltrafastTwoPhotonImaging,Abdelfattah2019BrightPhotostableChemigenetic}.
    }
    \label{fig:VI-sigs}
\end{figure}

Voltage imaging is very similar to calcium imaging: all or a genetically selected subset of neurons are made fluorescent, and these are scanned with a focused laser, to yield three-dimensional movies of neural activity.\cite{Knopfel2019OpticalVoltageImaging}
The difference is that the indicator molecules used in voltage imaging fluoresce in direct proportion to the membrane potential of the cell, instead of its calcium concentration. This then allows to directly observe the membrane potential of all neurons of interest in the field of view (\cref{fig:VI-sigs}).

Although voltage imaging has existed for a long time, the recorded signal has long been too weak to distinguish it from background noise (unless animals with very large neurons are used, or the activity of many co-firing neurons is pooled together).\footnote{
    One reason for the relative noisiness of voltage imaging is that the indicator molecules are embedded in the cell membrane, whereas in calcium imaging, the indicator molecules float around the entire cell \emph{volume}. The dendrites and axons of a neuron make up most of the cell's surface area, whereas the soma makes up most of its volume. Calcium indicators are thus much more localized than voltage indicators.
}
In recent years however, multiple labs have been iteratively refining the voltage indicator molecules. Together with the improvements in fluorescence imaging technology, driven by calcium imaging, this has made voltage imaging now powerful enough to image multiple individual neurons in vivo in common model animals. The signal-to-noise ratio has improved to the point that not only individual spikes, but also subthreshold voltage fluctuations can be observed.\footnote{
    \Cref{sec:voltage-imaging} further down expands further on the current capabilities of voltage imaging technology.
}
As explained next, it is precisely this level of detail that we believe enables in vivo connection mapping.


\section{Inferring wiring from activity}

\begin{figure}
    \vspace*{2em}  % space from vi fig above
    \hspace*{-2.5em}
    \includegraphics[w=1.2]{diagram_Connectivity↔Activity.png}
    \captionn
        {The causal link between neural connectivity and activity}
        {On the left, a cartoon of a synaptic connection. The axon of presynaptic neuron $M$ (in blue) impinges on postsynaptic neuron $N$ (in brown). The electrode icons indicate that their membrane voltages are recorded (shown on the right).
        A succesful spike in neuron $M$ will elicit a small but precisely-timed voltage bump in neuron $N$ (the postsynaptic potential, PSP).
        There is thus a causal relationship between 1) the existence of a connection $M$~→~$N$ and 2) both neurons' membrane voltages. This causal relationship (black arrow) is exploited to perform network inference from voltage recordings (green arrow).\newline
        \small{\emph{Drawings adapted from Purves et al.'s ``Neuroscience'' textbook, 6th edition, 2018.}}}
    \label{fig:diagram_Connectivity-Activity}
\end{figure}

The potential to infer the wiring from neural activity rests on the basic link between the two (\cref{fig:diagram_Connectivity-Activity}): an excitatory neuron that sends a spike will slightly increase the voltage of all its downstream neurons (this small increase is called the excitatory postsynaptic potential, or EPSP). When a neuron has received enough spikes, its voltage crosses a threshold, and it will send a spike itself. To estimate neural wiring, the idea is then to invert this reasoning: if neuron B often shows activity right after neuron A has fired, then neuron A is likely to be connected to neuron B.

As both calcium imaging and extracellular electrode recordings yield (at best) spike timing data only, existing activity-to-wiring approaches have been based only on spike timing\cite{MagransdeAbril2018ConnectivityInferenceNeural,Casadiego2018InferringNetworkConnectivity,Das2020SystematicErrorsConnectivity}, and not on more detailed measurements of neural activity. The problem with this is that the correlation between two neurons being connected and them spiking together close in time is quite tenuous. For one, most neurons need to receive many spikes – each of which can come from any of its hundreds to tens of thousands of input neurons – before it fires a spike itself. Second, many neurons have long time constants, meaning that a spike can influence spiking in its receiving neurons up to hundreds of milliseconds later.

As a result, spike-based wiring inference methods require long recording durations to obtain some confidence on the wiring between even small numbers of neurons.\footnote{
    E.g. in \cite{Orlandi2017FirstConnectomicsChallenge}, one-hour long recordings were used. This was long enough for a lot of the spike-based methods to recover edges, but \emph{not} the directionality of connections.
}
During these long recordings, the connectivity may have already changed. And long recordings are not possible for fluorescence imaging, as dyes require recovery after each recording session.

When we can observe the subthreshold increases in voltage occurring directly after each spike however, we might be able to accurately reconstruct connectivity from recordings on the timescale of individual in vivo experiments. The recent advances in voltage imaging provide exactly this kind of data.



\clearpage
\section{Voltage imaging in more detail}
\label{sec:voltage-imaging}

What follows is a short literature review of the current capabilities of voltage imaging technology. The main goal is to be able to build simulations in this thesis relevant to reality.

\FloatBarrier

\begin{figure}
    \includegraphics[w=1.5]{VI-cranial-window}
    \captionn
        {Camera view in voltage imaging}
        {Fluorescent neurons visible through a cranial window. This is just one imaging plane (i.e. there are more neurons visible above and below this plane). Voltage imaging recordings are thus 3D videos of neural membrane voltages.
        Scalebars 1 mm and 0.1 mm, respectively.
        Adapted from \cite{Abdelfattah2019BrightPhotostableChemigenetic,Knopfel2019OpticalVoltageImaging}.}
    \label{fig:VI-cranial-window}
\end{figure}

\begin{figure}
    \includegraphics[w=1.5]{VI-sigs-and-setups}
    \caption
        {Left: More example VI traces, with different spike-signal-to-noise ratios. Colours correspond to neurons with coloured numbers in \cref{fig:VI-cranial-window}. Right: in-vivo imaging setups.\\
        Adapted from \cite{Abdelfattah2019BrightPhotostableChemigenetic,Knopfel2019OpticalVoltageImaging}.}
    \label{fig:VI-sigs-and-setups}
\end{figure}



\subsection{Working principle}

Voltage imaging is a type of fluorescence microscopy, itself a type of light microscopy. A light source such as a laser or an LED shines on the imaged object -- such as a slice of brain tissue, a young transparent zebrafish, or part of a mouse brain made accessible through a hole in the skull. Most light passes through the sample of interest (brain tissue is mostly transparent for the used wavelengths). However, some parts of the sample 'reflect' light back.

These reflecting parts are so called voltage indicator molecules. Such molecules are introduced by the experimenter and gather spontaneously in cell membranes. The amount by which they reflect light back depends on the voltage placed over them. By capturing the reflected light from a sample, repeatedly over time, we thus get a movie showing 1) where cell membranes are, and 2) how the voltages over them change over time.

\subsubsection{Physical detail}
The above is a simplified description of the mechanics. This section gives a slightly more detailed description of what happens.

The actual mechanism of light 'reflection' is fluorescence, in which molecules absorb a photon, hold on to it for about one nanosecond, and then re-emit it again, at a longer wavelength. This process is about a million times slower than 'true' reflection (that is, photon scattering), but still a million times faster than the timescales we are interested in (milliseconds, corresponding to the duration of an action potential) and at which we image (with movie frame rates of 500 - 1500 Hz).\cite{Valeur2012MolecularFluorescencePrinciples,Cox2019FundamentalsFluorescenceImaging}

Furthermore, a change in voltage does not necessarily lessen the total fluorescence. Rather, the emission spectrum may shift. If you however only look at a fixed narrow band of the emitted light (as is done in fluorescence microscopy), the measured light will indeed seem to decrease or increase on voltage changes.
'Instantaneous reflection modulated by voltage' is thus a sufficient mental model for our purpose.

\subsubsection{Genetic targeting}

Since the late 90's, cells have been coerced into creating (parts of) indicator molecules themselves, by delivering transgenes into the cell.\cite{Siegel1997GeneticallyEncodedOptical}
This has multiple advantages. Most importantly, it allows the indicator molecules to be constrained to only certain cell types, by placing the transgene under the control of promotors that are only active in the cell type of interest. By doing so, scientists can avoid labelling glial cells. This decreases the background signal. Going further with the same principle, they can selectively label one type of neuron (for example, only interneurons in one hippocampal area).\cite{Hochbaum2014AllopticalElectrophysiologyMammaliana}

Another advantage, also decreasing the background signal, is that indicator molecules can be constrained to the membrane of only the cell body, and not the membranes of dendrites and axon branches. This is done by adding small cell-body-targeting signalling sequences to the transgene.



\subsection{System performance}

The history of voltage imaging has mostly been the history of finding better indicator molecules. Earlier versions (starting from their invention in the late 60's) changed their reflectance only weakly and slowly in response to a change in voltage. In addition, they were toxic, disrupting the normal functioning of cells, or altogether destroying them.\cite{Salzberg1973OpticalRecordingImpulses,Zochowski2000ImagingMembranePotential}
Modern indicators are no longer toxic, and their voltage sensitivity and speed have improved substantially (by more than ten- and hundredfold, respectively).\cite{Miller2016SmallMoleculeFluorescent,Tomina2018DualsidedVoltagesensitiveDye}


\subsubsection{Fidelity}

Current voltage indicators can have sub-millisecond time constants, meaning they can replicate the shape of an action potential as well as an electrode recording. During an action potential in a brain slice, the measured fluorescence of current voltage indicators increases by 30\% (± about 20\%)\cite{Piatkevich2019PopulationImagingNeural,Adam2019VoltageImagingOptogenetics}\footnote{
    Round brackets in this section indicate a rough indication of variation between experimental setups. See \href{https://docs.google.com/spreadsheets/d/1W9Y3az4i1xdvahpdyqtsTG8F81LXK2T6wzRgsXHN3z0/edit}{this spreadsheet} for concrete data from some representative VI studies.
}
relative to baseline. In live, head-fixed mice, this fluorescence increase is about 8\% (± 2\%).\cite{Abdelfattah2019BrightPhotostableChemigenetic,Villette2019UltrafastTwoPhotonImaging}

More important is the change in brightness with respect to noise. For voltage imaging, a signal-to-noise ratio (SNR) is often defined as the height of the fluorescence signal during an action potential, divided by the standard deviation of the baseline fluorescence signal.\footnote{
    Note that such a definition is not compatible with the standard definition of SNR, where the denominator is most often the squared standard deviation of noise, and the numerator is the averaged squared value of \emph{all} samples, of some denoised `signal' time series.
}
The action-potential-SNR is around 30 (± 7) in brain slices\cite{Adam2019VoltageImagingOptogenetics,Piatkevich2019PopulationImagingNeural}, and around 10 (± 3) in live, head-fixed mice.\cite{Abdelfattah2019BrightPhotostableChemigenetic,Villette2019UltrafastTwoPhotonImaging}

There has been no quantification yet of how well voltage indicators track subthreshold voltages. Estimating visually from published figures however, the correspondence between simultaneous electrode and optical recordings is substantial, and seems good enough to calculate with, even in recordings from live animals.


\subsubsection{Yield}

The number of simultaneously voltage-imaged neurons in live mice varies between 4 and 46 in the latest studies (with frame rates of mostly 500 Hz).\cite{Adam2019VoltageImagingOptogenetics,Villette2019UltrafastTwoPhotonImaging,Abdelfattah2019BrightPhotostableChemigenetic,Piatkevich2019PopulationImagingNeural}

For comparison, calcium imaging yields between 200 and 1000 neurons at a frame rate of 30 Hz, and up to 10,000 at 2 Hz.\cite{Pachitariu2017Suite2p10000} These higher yields come of course at the cost of less information per neuron: voltage imaging tracks subthreshold voltages and detects nearly all spikes, while calcium imaging, especially at such frame rates, yields only spike detections, imprecise both in time and in number.

The number of simultaneously imaged neurons is in any case bound to increase for both modalities, as microscopic scanning systems get faster.

Voltage imaging can only be performed in relatively short continuous bouts: the same mechanism that makes the molecules fluorescent -- namely, excitation on impact of a photon -- also makes the molecules more chemically reactive, making them spontaneously break down ('photobleach') in reaction with their environment. The total fraction of not-yet photobleached indicator molecules decays exponentially -- and so do the fluorescence and the spike SNR. The time constant of this decay (i.e. duration after which SNR has decreased by 63\%) is around 10 ± 5 minutes.\cite{Piatkevich2019PopulationImagingNeural,Abdelfattah2019BrightPhotostableChemigenetic} Most voltage imaging sessions are therefore not longer than this.

Cells replace broken indicator molecules relatively fast however, and experiments have shown that imaged photobleached cells show complete recovery after two days. Even shorter intervals between imaging sessions might be possible.

Because modern indicator molecules are not toxic, cells can be intermittently imaged over long periods -- one study followed the same neurons in vivo over more than a month.
% [can't find ref anymore]



\section{Conclusion}

In this first chapter, we described voltage imaging technology, and introduced the idea to use the recordings it generates to reconstruct neural connectivity.

Our aim in this thesis is to test whether this idea is feasible in principle: can we perform network inference from voltage signals? To test this, we will work with simulated data: We'll generate neural activity from a known connectivity, and develop and test different connection detection methods on this simulated data.

\Cref{ch2} describes the simulation: the choices we make for the used neuron model, synapse model, inputs, ..; and their parameters. Having set up our test environment, \cref{ch3-STA} introduces the first and simplest connection detection method, and different ways to evaluate its performance. Up to that point, only a simplified network is used (where $N$ Poisson inputs impinge upon a single simulated neuron). \Cref{ch4-network} expands the test to a full recurrent network of simulated neurons. Finally, in \cref{ch5-new-methods}, we introduce and evaluate three new methods for network inference based on voltage signals. \Cref{ch6} concludes the thesis with an overview of possible future research directions and a summary of our findings.

\clearpage

\section{Connectomics}

\begin{itemize}
    \item Motivation
    \item Ground-truth connectomics: tracing of electron microscopy and fluoresent injection imaging
    \item Necessity of connection \emph{inference}
    \item Mention `invasive' connection testing (stimulate one cell, record possible neighbours)
    \item Limitations of `connectomics', and of inferred vs `actual' connectomics.\\
        Terminology, e.g. `functional connectomics'
\end{itemize}


\section{Voltage imaging}

\begin{itemize}
    \item Technologies (from dyes to GEVIs)
    \item Specs: cell yield, tissue depth, recording duration, SNR, species
    \item ..and growth of these over time, and comparison with calcium imaging.\\
        To extrapolate how these might advance in the future
    \item Comparison with other recording techniques: ephys, calcium imaging, (and briefly mention coarser methods)
\end{itemize}


\section{Network inference}

\begin{itemize}
    \item Working with events/spikes only, versus working with continuous signals; or a hybrid as here.
    \item Overview of the spikes-only methods
    \item The connectomics competition, and the findings about the best methods
    \item Mention other application domains, like gene regulatory networks
    \item Conclusion: connection inference by spikes is poor (Ila Fiete etc), and Vm imaging offers unique advantage (causality)
    % \item 'Spike-triggered voltage regression' of Zhou/Cai
\end{itemize}



% \chapter{Connection tests \&\\ the N-to-1 experiment}

\chapter{Simulation details}

In this chapter, we decribe our experimental setup: the neuron model we simulate, its inputs, and how we simulate voltage imaging.


\section{The AdEx neuron model}

We choose to simulate the `AdEx' neuron model, or the `adaptive exponential integrate-and-fire' neuron.\cite{Gerstner2009AdaptiveExponentialIntegrateandfire}
This is a leaky-integrate-and-fire (LIF) neuron model, with two additions.
First, the full upstroke of each spike is simulated, as an exponential runoff.
Second, an extra dynamic variable is added: the adaptation current.
This allows the simulation of many non-linear effects of real neurons, like spike-rate adaptation and post-inhibitory rebound. (Note however that we do not focus on any of these effects in this thesis).

The AdEx model consists of two differential equations, one to simulate the membrane voltage $V$, and one for the adaptation current $w$:
\begin{align}
    C \d{V} &=  -g_L (V - E_L)
                            + g_L Δ_T \exp \left(\frac{V-V_T}{Δ_T}  \right)
                            - I_\syn - w
                            \label{eq:AdEx-V}
                            \\[1em]
    τ_w \d{w} &= a (V - E_L) - w
        \label{eq:AdEx-w}
\end{align}

$I_\syn$ is the synaptic current, explained in \cref{sec:synapse_model}.
We use the sign convention of inter alia Dayan \& Abbott (\cite{Dayan2001TheoreticalNeuroscienceComputational}, ch.~5.3, p.~162) where membrane currents are defined as positive when positive charges flow \emph{out} of the cell. I.e. a positive $I_\syn$ decreases the membrane voltage (itself defined as the electric potential inside minus outside the cell).

Other parameters are explained in \cref{tab:AdEx-params}.

In this chapter, we will often analyse $C\d{V}$ as a function of $V$, i.e. analyse it as a dynamical system: will the voltage increase or decrease at the current voltage?
For conciceness, we will call this function $F(V)$. I.e. $F(V) = C \d{V} =$ the right-hand-side of \cref{eq:AdEx-V}, here.

\begin{table}[h]
    \begin{sidecaption}
        {Quantities and parameters of the AdEx neuron, \cref{eq:AdEx-V,eq:AdEx-w,eq:AdEx-reset}. By defining the location of $F(V)$'s minmum, $V_T$ also determines the location of the firing threshold.}
        [tab:AdEx-params]
        \begin{tabular}{c l l}
            Name  & Description & Units \\
            \hline
            $V$  & Membrane voltage & V \\
            $g_L$ & Input / leak conductance & S (V/A) \\
            $E_L$  & Resting / leak potential & V \\
            $Δ_T$  & Steepness of $F(V)$ around the firing threshold & V \\
            $V_T$  & Location of minimum of $F(V)$ & V \\
            $w$   & Adaptation current & A \\
            $τ_w$   & Time constant of adaptation current & s \\
            $a$ & Sensitivity of adaptation current to $V$ & S \\
        \end{tabular}
    \end{sidecaption}
\end{table}

We solve these equations using first-order (Euler) integration.

In addition to the two differential equations, the AdEx model also consists of an instantaneous reset condition. When the membrane voltage $V$ reaches a certain threshold, a spike is recorded, $V$ is reset, and $w$ is increased:

\begin{align}
    \text{if}\ V > θ\ \text{then:}\ \label{eq:AdEx-reset}
    & V ← V_r \\
    & w ← w + Δw \notag
\end{align}


\section{Alternative neuron models}

Why did we choose the AdEx model to simulate neuron voltages? In short, because it strikes a good balance between realism and complexity. (Complexity, as in: still having a compact set of free parameters). We discuss here briefly two alternative neuron models: the simple leaky-integrate-and-fire (LIF) neuron, and the more complex Hodgkin-Huxley (HH) neuron.
In the next section, we go into more depth on a third alternative, the very similar Izhikevich neuron.

A simpler model than AdEx would be the well-known LIF neuron:
\begin{align*}
    C \d{V} =  -g_L (V - E_L) - I_\syn \\[1em]
    \text{if}\ V > θ\ \text{then:}\ V ← V_r \\
\end{align*}
As is apparent from comparing with \cref{eq:AdEx-V,eq:AdEx-reset}, the AdEx model is an extension of the LIF model. The LIF neuron lacks a simulation of the upstroke of spikes (the exponential term in \cref{eq:AdEx-V}), and the slower time-scale adaptation current (\cref{eq:AdEx-w}), which allows the simulation of many qualitatively different real neuron types.
It is especially this first addition, the full upstroke simulation, that seems relevant in generating realistic voltage traces.

Would this thesis have been very different had we used LIF neurons instead?
Probably not, though it might depend on the mean voltage level of the simulated neuron: if it is well below the firing threshold, both LIF and AdEx are linear (the exponential term is negligible), and they behave quasi identically. When a spike is generated in the AdEx model, the exponential feedback makes the upstroke very fast, and thus not many timesteps in the simulation are spent on it, versus the linear regime.
[Reference the Vdot-V, dynamical system fig]

On the other hand, when the neuron would continuously teeter just below its firing threshold, the LIF and AdEx models do not behave similarly. LIF's $F(V)$ curve is still fully linear, while AdEx's is not, and AdEx will behave more like a real neuron. [Reference fig with real neuron Vdot-V curve]

Another well-known alternative neuron model is the class of Hodgin-Huxley (HH)-like neurons. These models simulate the full trajectory of a spike: both its upstroke and its downstroke. Unfortunately they also have many free parameters. They also take a bit longer to simulate, being higher dimensional (having more differential equations), and containing many more exponential terms, which take the brunt of the time when numerically evaluating a differential expression.


\section{The Izhikevich neuron}

These are the Izhikevich equations, using the same symbols as used before (in \cref{eq:AdEx-V,eq:AdEx-w,eq:AdEx-reset}):

\begin{align}
    C \d{V} &=  k (V - E_L) (V - E_T) - I_\syn - w  \label{eq:Izh-V}  \\[1em]
    τ_w \d{w} &= b (V - E_L) - w     \label{eq:Izh-w} \\[2em]
    \text{if}\ V& > θ,\ \text{then:}  \label{eq:Izh-reset} \\
        &V ← V_r \notag \\
        &w ← w + Δw \notag
\end{align}

We have introduced two new parameters not present in the AdEx equations: $k$, the steepness of the parabola; and $E_T$, the 'instantaneous' firing threshold -- that is, the firing threshold in the absence of any synaptic or adaptation currents.

Note that, in the absence of synaptic and adaptation currents, \cref{eq:Izh-V} is a quadratic function of $V$, and its two zeros, $E_L$ and $E_T$, are apparent.
These two zeros are respectively the stable fixed point (the resting threshold or leak potential $E_L$), and the unstable fixed point (the firing threshold $E_T$).


\subsection{Correspondences with AdEx}

In Izhikevich's book \cite{Izhikevich2007DynamicalSystemsNeuroscience} (section 5.2.4, equations 5.7 \& 5.8), different names are used for the same quantities:
\begin{align}
    C \d{v} &= k(v - v_r)(v-v_t) - u + I \label{eq:IzhIzh-v} \\[1em]
    \d{u} &= a(b(v-v_r) - u) \\[2em]
    \text{if}&\ v > v_\text{peak},\ \text{then:} \\
    v &← c  \notag \\
    u &← u + d  \notag
\end{align}

\Cref{tab:AdEx-Izh-compar} compares both notation conventions.

\begin{table}[h]
    \begin{sidecaption}
        {Different symbols used for the same quantities in \cite{Naud2008FiringPatternsAdaptive} (`AdEx') and \cite{Izhikevich2007DynamicalSystemsNeuroscience} (`Izh').
        Membrane capacitance $C$ (in farad) is the same in both notations.}
        [tab:AdEx-Izh-compar]
        \begin{tabular}{c c l c}
            AdEx   & Izh  & Description & Units \\
            \hline
            $V$  & $v$   & Membrane voltage & V \\
            $w$  & $u$   & Adaptation current & A \\
            $τ_w$  & $1/a$   & Time constant of adaptation current & s \\
            $E_L$  & $v_r$  & Resting / leak potential & V \\
            $V_r$  & $c$  & Reset voltage after spike & V \\
            $a$  & $b$ & Sensitivity of adapt. current to $V$ & S \\
            $b$  & $d$ & Adaptation current bump after spike & A \\
        \end{tabular}
    \end{sidecaption}
\end{table}

Beside these straightforward correspondences, there are some parameters in either model that have no direct equivalent in the other: $k$ and $E_T$ in Izhikevich, and $g_L$, $Δ_T$, and $V_T$ in AdEx.
Here, we find correspondences for those parameters too.

First, the AdEx parameter $g_L$. This is the input conductance, a.k.a. the leak conductance, a.k.a. the slope of $F(V)$ around the leak potential. We can find this same conductance for the Izhikevich neuron too, by taking the derivative with respect to $V$ of the right hand side of \cref{eq:Izh-V}, at $w = 0$, $I_\syn = 0$, and $V = E_L$. We find:
\begin{equation}
    \dd{V}(k(V-E_L)(V-E_T)) \Big|_{V=E_L} = k(E_L - E_T)
\end{equation}
(this value is negative: the leak potential is a stable fixed point. This corresponds to \cref{eq:AdEx-V}, where we find `$-g_L$'.).
We thus have our first nontrivial correspondence between the AdEx and Izhikevich parameters:
\begin{equation}
    g_L = k (E_T - E_L)
\end{equation}

Next, we look at the AdEx parameter $V_T$. As mentioned before, this is the minimum of ts the AdEx $F(V)$ function. We show this here by finding roots of the derivative of $F(V)$ (\cref{eq:AdEx-V} in the absence of synaptic and adaptation currents):
\begin{align}
    \d[V]{F} &= \dd{V} \left( -g_L (V - E_L)
         g_L Δ_T \exp\left(\frac{V - V_T}{Δ_T}\right) \right) \notag \\
    &= -g_L + g_L Δ_T \frac{1}{Δ_T} \exp\left(\frac{V - V_T}{Δ_T}\right)
\end{align}
We indeed find a zero of $\d[V]{F}$ at $V = V_T$.

I.e, unlike what is suggested by the `$T$' subscript, and the name `effective threshold potential' given to it in \cite{Naud2008FiringPatternsAdaptive}, $V_T$ is not the instantaneous threshold potential (it is not a zero of $F(V)$), but rather the minimum of $F(V)$ (it is the zero of $\d[V]{F}(V)$).

The minimum of Izhikevich's $F(V)$ is easily found as the average of the parabola's two zeros.
We thus find our second nontrivial correspondence between an Izhikevich and an AdEx neuron:
\begin{equation}
    V_T = (E_L + E_T) / 2
\end{equation}



\subsection{Comparison with AdEx}

These models are very similar. Their phase spaces are topologically identical:
the adaptive current equation is identical (up to a renaming of the variables); and the $\dot{V}(V)$-graph has the same shape, with two fixed points: a stable fixed point at the resting potential, and an unstable one at the firing threshold.

They differ in the exact shape: Izhikevich's $F(V)$ is a parabola, while AdEx is the more realistic 'linear subthreshold and then transitioning to an exponential'. [ref bio fig]
As a result, Izhikevich neurons have an unrealistically slow spike upstroke.


\clearpage
\subsection{Synapse model}
\label{sec:synapse_model}

One unexplained term in our neuron model \cref{eq:AdEx-V} is the synaptic current, $I_\syn$.
This is the following sum over all input synapses $i$ of the neuron:
\begin{equation}
    I_\syn = \sum_i g_i (V - E_i)
    \label{eq:I_syn}
\end{equation}
where $V$ is the global membrane voltage of the neuron, $E_i$ is the reversal potential of that synapse, and $g_i$ is the local synaptic conductance, which is modulated by presynaptic spikes.

For an excitatory synapse, $V < E_i$, making $g_i (V - E_i)$ negative, increasing the membrane voltage according to the sign convention for $I_\syn$ in \cref{eq:AdEx-V}.

We simulate the synaptic conductances $g_i$ as exponentially decaying signals (with time constant $\tau$), and bump them up instantaneously on arrival of a presynaptic spike:
\begin{gather}
    \d{g_i} = -g_i / \tau
    \label{eq:g_i}
    \\[1em]
    \text{On incoming presynaptic spike:}\notag\\
    \ g_i ← g_i + Δg_i
\end{gather}

\marginpar{
    \includegraphics{g1.pdf}
    \captionof{figure}{Example synaptic conductance trace $g_1(t)$, with a single incoming spike at $t = 20$ ms.}
    \label{fig:g1}
}
\marginpar{
    \vspace*{3em}
    \includegraphics{g2.pdf}
    \captionof{figure}{Another example trace $g_2(t)$, with spikes at $t = 10$ ms and $30$ ms, and a smaller $Δg$.}
    \label{fig:g2}
}

Note that these are not the so called alpha-synapses. Those are two dimensional and also have an exponential rise, instead of just an exponential decay. (For an infinitely fast rise though, these models are of course the same).
Simulating a full alpha synapse might increase the realism of our voltage traces, for a small simulation cost. We did not try this however. Foremost because alpha synapses fit to real data often have very fast rise times that are almost indistinguishable from instantaneous jumps.

For efficiency, we give all our excitatory synapses the same reversal potential, $E_\exc$. Idem for the inhibitory synapses, with $E_\inh$. This allows us to factor the synaptic current sum (\cref{eq:I_syn}) as follows:
\begin{equation}
    I_\syn = (V - E_\exc) \sum_{\exc\ i} g_i \  + \  (V - E_\inh) \sum_{\inh\ i} g_i
    \label{eq:I_syn_factor}
\end{equation}

The sums of conductance signals $g_i(t)$ can also be simplified. Say that the values of $g_i$ at $t = 0$ are $G_i$. The solution to \cref{eq:g_i} (at least in the time until a new presynaptic spike arrives) is then
\begin{equation}
    g_i(t) = G_i\ e^{-t/\tau}
\end{equation}

With this, and when all synapses have the same time constant $\tau$, the two sums in \cref{eq:I_syn_factor} can be factored as follows:\footnotemark
\footnotetext{
    This is only valid in the time before any new spikes arrive.
    To see that the 'summability' still holds after a new spike arrives, the above reasoning can be repeated, but simply with different values for the $G_i$ (all decayed by an amount $e^{-t_\text{spike}/\tau}$, and one increased by a bump $Δg_i$), and then redefining $t_\text{spike}$ to be $t = 0$.
}
\begin{equation}
    \sum_i g_i(t) = \sum_i \left( G_i\ e^{-t/\tau} \right)
                            = \left( \sum_i G_i  \right) e^{-t/\tau}
\end{equation}
This means that we need to only keep track of two conductance signals: $g_\exc$ and $g_\inh$, each the sum of all excitatory or all inhibitory synaptic conductances.

\marginpar{
    \vspace*{2em}
    \includegraphics{g3.pdf}
    \captionof{figure}{A third synaptic conductance trace $g_3(t)$, with three input spikes at the same times and strengths as in \cref{fig:g1,fig:g2}. This signal is simulated independently, but turns out to be equal to the sum of the two others: $g_3(t) ≡ g_1(t) + g_2(t)$.}
}

Our synaptic current sum then becomes simply:
\begin{equation}
    I_\syn = (V - E_\exc)\ g_\exc \  + \  (V - E_\inh)\ g_\inh,
\end{equation}
and we only need to simulate two differential equations, instead of one for every synapse:
\begin{align*}
    \d{g_\exc} &= -g_\exc / \tau \\
    \d{g_\inh} &= -g_\inh / \tau,
\end{align*}
where on arrival of a spike at synapse $i$ either $g_\exc$ or $g_\inh$ is instantaneously increased by a value $Δg_i$, depending on whether that synapse is excitatory or inhibitory.


\section{Input spikes}

In our simplest experimental setup, we simulate just one AdEx neuron.
Its input is provided by an array of $N$ Poisson neurons, i.e. they each generate spike trains according to a Poisson process. We call this the `N-to-1' setup.

The inter-event intervals of a Poisson process follow an exponential distribution.
We use that fact to generate spike trains: we draw samples from $\mathrm{Exp}(\lambda)$ (with $\lambda$ the desired firing rate), and cumulatively sum up these intervals  to obtain spike times. This is done until we have reached the desired input train duration.


\section{Voltage imaging}

The signals detected by a light microscope in a voltage imaging setup are not the same as the real membrane voltage signals of which they are a reflection.

We model this lossy transformation by simply adding Gaussian noise to our simulated membrane voltage. As in the voltage imaging literature, we quantify the amount of this  noise by a `spike-SNR' measure (spike signal-to-noise ratio). This is defined as the height of an average spike relative to the standard deviation of the noise.

A more realistic model of the voltage-imaging transformation would also incorporate the exponential decay over time of the SNR, and the short-term 'smearing in time' of voltage indicators. The latter could be done by passing the voltage signal through a linear filter with some non-instantaneous impulse response.


% TODO here:
% - Make an appendix, put in the math derivations of Izh and AdEx (and the comparison table). In main text, only have all current figs (F(V), slow rampup, nonlinear subthr), and a summary of the text reasons for Izh→AdEx. (and add a little historical note: most of the phd done with izh, so there was some switching cost, so we need to justify the switch well :)).

% \begin{itemize}
%     \item AdEx neuron, synapse model, exc and inh inputs
%     \item Poisson spike generation
%     \item Lognormal distribution of input firing rates
%     \item Influence of N \& input strength, on output firing rate \& average voltage level
%     % \item Evaluation plots
% \end{itemize}


\chapter{Spike-triggered averaging}
\label{ch:STA}


As shown in \cref{fig:diagram_Connectivity-Activity}, our idea for connection inference rests on the causal link `presynaptic spike' → `postsynaptic voltage bump'. I.e. we want to know for which neuron pairs a spike in one is reliably followed by such a bump in the other. The problem is that these bumps (the postsynaptic potentials or PSPs) are minute, and are easily drowned out by (1) other PSPs, (2) postsynaptic spikes, and (3) voltage imaging noise.

So, as is often done in neuroscience, we take the \emph{average} over many instantiations, so as to hopefully find a signal in the noise. Specifically, we take spike-triggered averages, or \textbf{STA}s, of neurons' voltage traces. If there is a connection from a neuron `M' to a neuron `N', then an STA of neuron N's voltage imaging signal, based on neuron M's spikes, would hopefully show the PSP.

And indeed, when we construct a few such STAs, we do see something resembling a PSP bump: \cref{fig:example_STAs}. We also find that the higher the firing rate of the presynaptic neuron, the cleaner the PSP-like shape is. This is of course because there are more presynaptic spikes and thus more windows to average over, which decreases the noise on the result. Finally, we see that inhibitory inputs result in downwards bumps in their STA, and excitatory inputs in upwards bumps.

\begin{figure}
    \hspace*{-3em}
    \includegraphics{example_STAs}
    \vspace*{1em}
    \captionn
        {Example STAs in the 10' simulation with 6500 inputs}
        {
        Note that every panel has a different y-axis (voltage) scale. The STA of the most active input is repeated in every panel (in faded blue), to allow a visual scale comparison nonetheless.\\
        The inset legends indicate with how many presynaptic spikes the STA was calculated, and whether the input was an excitatory or inhibitory one.\\
        The top right panel shows STAs of the 1\ts{st} and 100\ts{th} fastest spiking inputs, both within the excitatory inputs (\mpl{blue} shades), and within the inhibitory inputs (\mpl{orange} shades).\\
        Source: \nburl{2023-09-13__Clippin_and_Ceilin}.
        }
    \label{fig:example_STAs}
\end{figure}

Note that in this chapter -- and the next one -- we only look at the so called N-to-1 case (\cref{fig:diagram_Nto1}), where we simulate the voltage of one neuron, impinged on by N independent Poisson spiketrains. This is done for simplicity; it is only in the "Networks" chapter later on that we look at full networks, were inputs might be correlated with one another.

\begin{figure}
    \vspace*{2em}
    \hspace*{-1em}
    \includegraphics[w=1.2]{diagram_Nto1.png}
    \vspace*{-1.4em}
    \captionn
        {The `N-to-1' problem}
        {\Left: A neuron $N$ (orange circle), and the spike trains of other neurons in the network (blue). Some of these other neurons impinge directly on $N$ (black arrows), while others are not (directly) connected (dashed gray lines). Given only neuron $N$'s voltage signal and the other neurons' spiketrains, we want to detect the direct inputs, while rejecting the not-directly-connected spiketrains.\newline
        \Right: The simulated membrane voltage of the impinged-upon neuron (orange), and the same signal with Gaussian noise added, to simulate a voltage imaging signal (blue). Underneath the plot, one of the possible input spiketrains, time-aligned to the voltage signal.
        This alignment is used later to extract spike-triggered windows from the voltage signal.}
    \label{fig:diagram_Nto1}
\end{figure}

To use spike-triggered-averages as an actual connection test, we look specifically at the height of an STA, and compare it to a distribution of STA heights that we'd expect were the two neurons not connected. This is illustrated and explained in more detail in \cref{fig:STA-height-suffle}. This so called 'shuffle' test yields the proportion $p$ of how many shuffled (random) spiketrains yield an STA with a larger height than the real STA. In a following section (\nameref{sec:perf_quant}), we'll use this number (as $t = 1 - p$) to make predictions and compare them to the ground truth.

\begin{figure}
    \hspace{-5em}
    \includegraphics[w=1.7]{diagram_STA_test.png}
    \captionn
        {A simple connection test: STA height with shuffle control}
        {The spikes of a possible input neuron are aligned to the voltage trace of the neuron of interest $N$, as in \cref{fig:diagram_Nto1}. For every such spike, a 100-ms long window is cut out of the voltage of $N$. The average of all these windows is called the spike-triggered average (STA).\newline
        \Left: Two example STAs of neuron $N$'s membrane voltage: one for an actually connected input neuron, $M$ (top, orange); and one for a non-input neuron (below, gray).
        Given an STA signal $x$, we will use its height $h = \max(x) - \min(x)$ (also known as `peak-to-peak' or `ptp') to test whether two neurons are connected. \newline
        \Right: An STA of $N$'s membrane voltage using a shuffled version of $M$'s spike times (which is made by randomly permuting the inter-spike-intervals of $M$). This `shuffled STA height' provides a control for the STA height connection test statistic: "what do we expect the STA height to be if there is \emph{no} connection $M$→$N$".
        By calculating different such shuffles, we obtain a null-distribution for the STA height test statistic. And by comparing the real STA height to this distribution, we can calculate a $p$-value. Here, the real STA is larger than all shuffle controls, of which there are 100. So $p < 0.01$, and at α = 0.05, we conclude there is indeed a connection $M$→$N$.}
    \label{fig:STA-height-suffle}
\end{figure}



\FloatBarrier
\section{Ceiling and clipping}
\label{sec:ceil-n-clip}

\begin{figure}
    \includegraphics{ceil_n_clip__sigs_and_STAs}
    \caption
        {Example voltage traces and corresponding STAs, where the only difference is the height of spikes. In blue, the unmodified simulated voltage trace. In orange, the same, but with ceiled spikes (as in \cref{sec:spike_ceiling}). In green, the same as orange, but with the spikes clipped again after the ceiling (as explained in this section).\\
        Source: \nburl{2023-09-13__Clippin_and_Ceilin}.}
    \label{fig:ceil_n_clip__sigs_and_STAs}
\end{figure}

As explained in \cref{sec:spike_ceiling}, we modify our simulated voltage trace so that spikes have a consistent height. This modification has an effect on STAs, as is illustrated in \cref{fig:ceil_n_clip__sigs_and_STAs}: the blue trace is the signal without spike ceiling, the orange one with spike ceiling. Their corresponding STAs are shown on the right. Note that the orange STA (made with ceiled spikes) is much noisier than the blue STA (from the unmodified voltage trace).

This suggests a relatively easy intervention to drasticaly de-noise STAs, and presumably increase their effectiveness for network inference: namely to remove the spikes from the signal.

We tried this 'spike clipping' and it indeed drastically denoised the STA; see the green signal and STA in \cref{fig:ceil_n_clip__sigs_and_STAs}.
We show that this decreased noise in the STA does indeed lead to an increase in network inference performance, by running a connection detection test without and with this spike clipping. The results are shown in \cref{fig:ceil_n_clip_AUCs}: detection performance increases from an AUC of 0.56 for the non-clipped voltage trace, to an AUC of 0.79 for the voltage trace with clipped spikes.

\begin{figure}
    \begin{sidecaption}
        {
            Connection detection performance for the three ways of handling spikes shown in \cref{fig:ceil_n_clip__sigs_and_STAs}: not modifying them; ceiling them; and clipping them.
            The area-under-the-curve or AUC measure is explained in the next section.\\
            Source: \nburl{2023-09-13__Clippin_and_Ceilin}.
        }
        [fig:ceil_n_clip_AUCs]
        \includegraphics[w=1.03]{ceil_n_clip_AUCs}
    \end{sidecaption}
\end{figure}
% TODO: this graph is all wrong:
%   - The colours are confusing (should be organized the other way round: colours are sig type, grouping is exc/inh/both).
%   - The random chance line is much lower, as shown below.
%
% Where to edit this?
% can redo in new nb :)


\FloatBarrier
\section{Performance quantification}
\label{sec:perf_quant}

It is not easy to express in a single number how good a network inference algorithm is. Depending on what you find important as a user, different measures make more sense than others. This section looks at some measures to quantify the performance of our algorithms, and discusses the merits and disadvantages of each.

All the algorithms that we look at in this and the following chapter eventually output a single number per tested neuron pair (A, B): "How strongly do I believe there is a connection A→B?". (And: "Is that connection excitatory or inhibitory?": the sign of the number). We will call this connected-ness number "$t$".

To get actual predictions out of the algorithm,
% ("excitatory connection", "inhibitory connection", "unconnected")
we must apply a threshold $θ$ to these measures. If $|t| > θ$, we classify the pair as connected, and as unconnected otherwise. For the detected pairs, we classify them as excitatory if $t$ is positive, and inhibitory if it is negative.

Note that we use the same threshold for both excitatory and inhibitory connections. We could in fact use a different threshold -- and we briefly look at this in \cref{fig:perfmeasures_threshold_PPVs_EI}) -- but for simplicity, we apply the same threshold for both types of connection.

Each threshold chosen yields a different tradeoff between recall and precision (\cref{fig:perfmeasures_θ_TPR_ROC}). At low thresholds, we can detect more connections ("true positives"); but we will also detect more non-inputs as being connected (false positives). This also lowers our precision.\\
Some definitions:
\begin{itemize}
    \item True positive rate (TPR), aka recall, sensitivity, and power: out of all true connections tested, how many did we correctly classify?
    \item False positive rate (FPR): out of all distractors we added to our test (randomly generated spiketrains), how many did we wrongly classify as an actual input?
    \item Precision, aka positive predictive value (PPV): out of all the neuron pairs that we classified as connected, how many are actually connected?
\end{itemize}

There are more measures that quantify the performance of a binary classifier at a given threshold than those three (such as negative predictive value, false discovery rate, false omission rate, ...).\footnotemark{}
But recall, FPR, and precision are commonly used ones.
\footnotetext{We do not actually perform binary classification: there are three classes (excitatory, inhibitory, unconnected). But it is not pure ternary (multiclass) classification either: we first classify as connected or not, and then (for the connected ones only), as excitatory or inhibitory. We could thus call it some kind of nested binary classification.}

\begin{figure}
    \includegraphics{perfmeasures_θ_TPR_ROC}
    \caption
    {Lower detection thresholds increase both true and false positives. On the right, true positive rates are plotted against the false positive rate, to obtain the so called receiver operating characteric or ROC curve. As the FPR increases more or less linearly with the decrasing detection threshold, both graphs look very similar.
    Source: \nburl{2023-09-13__Clippin_and_Ceilin}.}
    \label{fig:perfmeasures_θ_TPR_ROC}
\end{figure}

Note that true positive rate (TPR) and precision are similar, in that they both count correct classifications. (They both have the number of true positives in their numerator). But recall (TPR) looks at the number of true positives from the point of view of the ground truth (how many did we find), and precision looks at it from the point of the experimenter (out of what this algorithm gives us, how much is correct?).

False positive rate and precision are also similar, in that they both measure the number of false positives.\
One advantage of using FPR over precision though, is that FPR does not depend on the number of distractors (unconnected spiketrains) that we add to our tests.\footnote{Besides that, the more distractors we test, the more accurate our estimate of the FPR will be.}
Whereas we can arbitrarily increase precision by including less unconnected trains in our test -- up to the limit of 100\% precision, when we do not add any distractors and all tested trains are actually connected.

In our tests, we choose the number of unconnected trains rather arbitrarily. (For example, when we test 100 excitatory and 100 inhibitory inputs, we also generate and test 100 unconnected spiketrains). A better way to choose this number of distractors might be to estimate what a realistic fraction of unconnected neurons would be in a typical voltage imaging experiment. Given some patch of brain tissue and one of the neurons in it, how many of the other recorded neurons in that patch will be connected to it? This is an interesting research question -- and it is likely that answers can be found in the literature -- but we do not explore it here.

In \cref{fig:perfmeasures_θ_TPR_ROC}, we have looked at TPR and FPR, both as a function of the detection threshold and as a function of each other. In \cref{fig:perfmeasures_Fscores,fig:PR_curves_iso_Fβ,fig:perfmeasures_PR_curves_EI}, we look at TPR (recall) and precision, again as a function of the detection threshold and as a function of each other.

Because recall and precision both increase for 'better' detectors, we might combine them into one measure. This is what the $F$-scores do: they are the harmonic mean of recall and precision, with recall and precision weighted differently depending on a parameter $β$. The $F_β$ score attaches $β$ times as much weight to precision $P$ as to recall $R$:
\begin{equation}
    F_β = \frac{(1+β^2) · P · R}{β^2 · P + R}
\end{equation}
For $β = 1$, precision and recall are weighted equally. The $F_1$-score is also the most widely used of the $F$-scores.

% \rule[0.5ex]{4.5in}{0.55pt}

\begin{figure}
    \includegraphics{perfmeasures_Fscores}
    \captionn
        {Lower detection thresholds trade-off higher recall for lower precision}
        {The $F_β$ scores interpolate between the two measures. $F_{→∞}$ is recall, $F_{→0}$ is precision. The black circles on the $F$-curves indicate their maxima. Different trade-offs between precision and recall (different $β$-values) thus dictate different optimal detection thresholds.}
    \label{fig:perfmeasures_Fscores}
\end{figure}

When we plot recall against precision, we get the so called PR-curves, shown in \cref{fig:PR_curves_iso_Fβ} for both excitatory and inhibitory inputs together, and in \cref{fig:perfmeasures_PR_curves_EI} for both types separately.
% Different $F_β$-scores (for the same $β$-value) correspond to the same rational function but with different offsetts. Different $β$-values on the other hand correspond to different scalings of the rational function.

\begin{figure}
    \includegraphics{PR_curves_iso_Fβ}
    \caption{Precision plotted against recall for the STA-test in the N=6500 inputs, 10-minute-recording experiment. Black dots indicate where three different $F_β$-scores reach their respective maximum values.}
    \label{fig:PR_curves_iso_Fβ}
\end{figure}

\begin{figure}
    \includegraphics{perfmeasures_PR_curves_EI}
    \caption{Same as in \cref{fig:PR_curves_iso_Fβ}, but with excitatory and inhibitory inputs analysed separately. Note that we can detect more inhibitory inputs than excitatory inputs for the same precision value (or for the same false positive rate, as shown in \cref{fig:perfmeasures_θ_TPR_ROC}).}
    \label{fig:perfmeasures_PR_curves_EI}
\end{figure}

\begin{figure}
    \begin{sidecaption}
        {\textbf{Excitatory and inhibitory inputs reach \maxF at different thresholds}.\\
        But for simplicity, whenever we use \maxF to evaluate a classifier, we will use only one threshold for both types of inputs, which will be a compromise between these two thresholds.}
        [fig:perfmeasures_threshold_PPVs_EI]
        \includegraphics{perfmeasures_threshold_PPVs_EI}
    \end{sidecaption}
\end{figure}

Different thresholds yield different $F_β$-scores; but there is one threshold where your chosen $F$-score is maximal. This \maxF-score is a good candidate for the single "how good is this detector" measure we were looking for. We specifically choose $F_1$ as it weighs precision and recall equally (and we have no a-priori reason to prefer any one over the other), and because it is the most frequently used.

Another common single measure to quantify a classifier's performance is the area under its ROC-curve, or AUC, already shown in \cref{fig:perfmeasures_θ_TPR_ROC}. A disadvantage of the AUC is that it is less interpretable as a number than the \maxF score (which immediately gives a rough idea of how many true connections you'll detect, and how many of your detections are correct).
An advantage of the AUC however is that it

\begin{figure}
    \begin{sidecaption}
        {\textbf{Randomly classifying connections yields a chance level AUC $< 0.5$}.\\
        300 random test results, and their performance as connection detector, quantified as area under their ROC curves. Solid black line is the mean, dashed line is the median.\\
        In every of the 300 simulations, every connection (100 exc, 100 inh, and 100 unconnecteds) was assigned a random `t-value' uniformly between $-1$ and $1$; and then the classification threshold was swept over these t-values.}
        [fig:AUC_chance_level]
        \includegraphics{AUC_chance_level}
    \end{sidecaption}
\end{figure}



\FloatBarrier
\section{Recording duration \& noise}

In this section we look at how longer or less noisy voltage imaging recordings improve connection inference. In \cref{fig:STA_perf_diff_snr}, the signal-to-noise ratio (SNR) is varied, and in \cref{fig:STA_perf_diff_rec_duration}, we vary the recording duration.

As might be expected, noisier and/or shorter recordings decrease detection performance, down to chance level in the limit (for noise almost as high as the spikes, and for recordings shorter than a minute). As to recording duration, interestingly, we do not yet see any flattening off of the detection performance curve for longer recording duratoins, up to the durations that we simulated (up to 1 hour).

\begin{figure}
    \begin{sidecaption}
        {\textbf{Performance drops to chance level for noisier signals}.\\
        All simulations were 10 minutes long.
        Signal-to-noise (SNR) values on the x-axis are approximately (but not exactly) log-spaced. An SNR of `$\infty$' corresponds to no noise (i.e. the voltage signal straight out of the simulation, without any noise added).
        For every SNR value, five different simulations were run (gray dots), each with a different RNG seed for input firing rate and spiketrain generation. The mean performances of these five simulations are plotted with larger dots and are line-connected.
        Only the 100 highest-firing excitatory and inhibitory inputs were tested. An additional 100 unconnected spiketrains were generated and tested, with similar firing rates as those 200 high-firing real inputs.\\
        AUC chance level determined as in \cref{fig:AUC_chance_level}.\\
        Source: \nburl{2023-09-20__STA_conntest_for_diff_recording_quality_n_durations}.}
        [fig:STA_perf_diff_snr]
        \includegraphics{STA_perf_diff_snr}
    \end{sidecaption}
\end{figure}

\begin{figure}
    \begin{sidecaption}
        {\textbf{Longer recordings allow more accurate connection inference}.\\
        All simulations had a voltage imaging noise level (spike-SNR) of 40.\\
        The simulation (`recording') durations are on a logarithmic axis. (The first two data points are at 10 and 30 seconds; the last one is at 1 hour). \\
        For more, see \cref{fig:STA_perf_diff_snr}'s caption.}
        [fig:STA_perf_diff_rec_duration]
        \includegraphics{STA_perf_diff_rec_duration}
    \end{sidecaption}
\end{figure}

For a concrete example of what a neuroscientist might expect from the STA-based connection test, we find that for a 30-minute recording with an SNR of 40\footnote{These are realistic values for voltage imaging, albeit on the optimistic side.},
the maximum F1-score -- for the 200-highest firing inputs, and an additionally tested 100 unconnected spiketrains -- is about 70\%.

I.e, in the N-to-1 setup with 6500 inputs, for the 200 highest-firing of those inputs, we detect approximately 140 of them as being connected (±70\% recall). And of the spiketrains that we detect as inputs, about 70\% are correctly classified  (±70\% precision). I.e. 30\% of them are either excitatory connections classified as inhibitory and vice-versa; or they are unconnected, random spiketrains classified as real inputs.

The area under the TPR/FPR-curve (AUC) for this recording duration and quality is about 0.65 (compared to the chance level of 0.25 -- see \cref{fig:AUC_chance_level}).



\FloatBarrier
\section{Computational cost of STA test}

\begin{figure}
    \begin{sidecaption}
        {\textbf{Test time scales linearly with voltage signal duration}.\\
        Simulation timestep ('sample time') of 0.1 ms. STA length of 20 ms; i.e. 200 samples.
        The chosen inputs to test (300 high firing trains) have a median firing rate of 16 Hz. I.e. at a simulation duration of 10 seconds, there are about 160 presynaptic spikes per tested connection. There are 101 times that many STAs to calculate per connection: once for the real spiketrain, and a 100 times for shuffles of it. For a 10 second simulation, there are thus about 16k STAs to calculate per connection.
        For 10 minutes: 967k STAs. For 1 hour: 5.8M STAs.\\
        Black dots are the means over five simulations per duration. Compute times for individual simulations are plotted with gray dots; but the variation is so small that these gray dots are hidden behind the black means. Gray dashed line is the $y = x$ identity.  Source: \nburl{2023-09-20__STA_conntest_for_diff_recording_quality_n_durations}.}
        [fig:STA_compute_time]
        \includegraphics{STA_compute_time}
    \end{sidecaption}
    % This cap too long. some info → broodtext
\end{figure}


% \begin{itemize}
%     \item Principle, example STAs
%     \item Influence on STA of E/I balance, output firing rate, reversal potentials
%     \item Use as connection test: shuffled spike trains and height of STA, p-values, and the `area-over-start' heuristic for E vs I classification.
%     \item Evaluation of a connection test: `ternary' classification, summary measures, AUROC
%     \item Performance of the simple STA-height test, for different N
%     \item Influence of window length
%     \item What about Vm imaging affects detection the most?
% \end{itemize}


\chapter{New connection inference methods}
\label{ch4}
% \begin{itemize}
%     \item Network connectivity, E/I balance, raster plots
%     \item Too many possible connections to test them all → Subsampling
%     \item Performance of last chapter's methods
%     \item If time: experiment with a network that is less densely connected than our current fully-random one. Why? To better examine the effect of indirect connections / colliders (For the current connectivity, there are too many of those. But in a more realistic, 'localized' network, there are less, and so it seems easier to isolate and examine their effect).
% \end{itemize}

\section{Introduction}

In the previous experiments, only one neuron's voltage was simulated. The inputs were Poisson spike trains.
In the next experiments, we simulate the voltages of a full network of neurons, which are recurrently connected to each other.
The goal is to investigate the effect on network inference of potentially correlated inputs and indirect connections.




\section{Connectivity structure}

\begin{figure}
    \subfloat{
        \includegraphics[w=0.75]{gephi-direct-inputs.png}
        \includegraphics[w=0.75]{gephi-inputs-to-inputs.png}
    }
    \vspace*{2em}
    \caption
        {\textbf{A neuron is reachable in two hops from most other neurons}.    Selected neurons and connections in our random network. Left shows the direct inputs to one of the neurons. Right additionally shows the direct inputs to these inputs. This subnetwork already contains more than 700 of the 1000 total neurons in the network.\\
        Excitatory neurons in green, inhibitory in red. The tangle in the middle consists of neurons that synapse onto multiple of the direct inputs of our selected neuron.\\
        Visualization using the `Gephi' software, with the `Yifan-Hu' layout algorithm, and default parameters otherwise.
        Source: \nburl{2022-08-29__Visualizing_subnets}.}
    \label{fig:gephi-network-viz}
\end{figure}

% These two margin figs can go to Appendix
\marginpar{
    \includegraphics[w=1, trim={0 0.4em 0 -1em}, clip]{shortest-path-1}
    \captionof{figure}{
        \textbf{A selected neuron is reachable in at most three hops.}
        Shortest path lengths from every other neuron in the network.}
    \label{fig:shortest-path-1}
}
\marginpar{
    \includegraphics[w=1, trim={0 0.4em 0 -1em}, clip]{shortest-path-all}
    \captionof{figure}{
        \textbf{Every neuron is reachable in at most three hops.}
        Shortest path lengths calculated using the Floyd-Warshall algorithm. Source: \nburl{2022-07-14__Unconnected-but-detected.html}.}
    \label{fig:shortest-path-all}
}

We choose the simple and common `fully random' connectivity rule,\footnote{
    Other common choices for connectivity structure are scale-free networks, and `local' networks.
}
where any neuron has a connection to another with a uniform random probability (we choose $p_\text{conn} = 0.04$). After generating an adjacency matrix this way (\verb|A = rand(N, N) .≤ 0.04|, where \verb|rand| draws from $\sim U[0,1]$), we remove autapses. We choose the number of neurons $N = 1000$.

A property of fully random networks is that they are strongly interconnected. In our network, any neuron is reachable from any other in at most three steps (three synapses); most are reachable in just two. This is exemplified in \cref{fig:gephi-network-viz} and \cref{fig:shortest-path-1}: one selected neuron (neuron `1' here) is reachable in two hops from more than 700 of the 1000 total neurons in the network. And when we compute the shortest path between every possible neuron pair, we find a very similar distribution (\cref{fig:shortest-path-all}).




\section{External input}

As we no longer have Poisson spike trains providing input to our neurons, we need another way of bootstrapping activity in the network.\\
Instead of external spikes, each neuron is provided with external input by adding Gaussian noise to its membrane voltage. Every time step ($Δt = 0.1$~ms), a sample drawn from a normal distribution with mean $–0.5$~pA and $σ = 5$~pA is added to the membrane current. (As membrane current is by convention negative, this corresponds to an on-average positive influence on membrane voltage).\\
Like in the first footnote in \cref{sec:model-voltage-imaging}, this way of adding noise makes the noise power dependent on the timestep. A more principled approach to injecting noise would be to replace our ODE for the neuron voltage (\cref{eq:AdEx-V}) by a stochastic differential equation (SDE); though we did not explore this here.


\section{EI balance}

Similar to the N-to-1 experiments, we make 1 out of 5 neurons inhibitory.
As before, this is done by setting the synaptic reversal potential at the outputs of inhibitory neurons to $-80$ mV (instead of the $0$ mV for excitatory neurons).
To make sure that each neuron receives a balanced mix of excitation and inhibition, and given that there are 4:1 excitatory to inhibitory neurons, we make excitatory neurons 4x weaker: their synaptic strength ($Δg$, the instantaneous increase in postsynaptic conductance $g$ on spike arrival) is 4x as small as that of inhibitory neurons.

% Option for appendix, these four figs
\begin{figure}
    \subfloat{
        \includegraphics[w=0.74]{1144_raster}
        \includegraphics[w=0.74]{1144_hist}
    }
    \captionn
        {Firing rates in the network with `1144' weights}
        {Left: rasterplot showing all spikes in the first 10 seconds of the simulation.\\
        Right: histogram of firing rates.\\
        Source: \nburl{2022-07-01__g_EI}.}
    \label{fig:1144-weights}
\end{figure}

\begin{figure}
    \subfloat{
        \includegraphics[w=0.74]{roxin_raster}
        \includegraphics[w=0.74]{roxin_hist}
    }
    \captionn
        {Firing rates in the network with `Roxin2011' weights}
        {See \cref{fig:1144-weights}.}
    \label{fig:roxin-weights}
\end{figure}

% A synaptic weight rule where inhibitory synapses are 4× stronger to compensate for there being 4× as many synapses.
% Given that we simulated four times as many excitatory as inhibitory neurons, the synaptic weight rule described above (with inhibitory synapses 4× as strong) seems logical, to keep excitation and inhibition in balance.
\Cref{fig:1144-weights} shows results of the network simulation as spiketrains and firing rate distributions, using the described synaptic weights rule (inhibitory synapses 4× stronger than excitatory ones). We find that the resulting firing rate distribution is fairly symmetrical, and that excitatory and inhibitory neurons have very similar firing rates.

In real neural networks however, inhibitory neurons often have higher firing rates than their excitatory neighbours. In addition, real firing rate distributions are often heavy-tailed and not symmetrical, as we saw back in \cref{sec:lognormal-poisson-input}.
So, to coax more realistic firing rate distributions out of our network, we looked at the previously mentioned modelling paper of Roxin \emph{et al}, ``On the Distribution of Firing Rates in Networks of Cortical Neurons'', \cite{Roxin2011DistributionFiringRates}.\footnote{
    This paper seeks to explain how heavy-tailed firing rate distributions emerge in randomly connected spiking neural networks. We wanted to emulate the simulations in this work, to obtain a heavy-tailed firing rate distribution.
}
% Their figures indeed seem to show a lognormal-like distribution of firing rates from their simulatd networks.
In this paper different synaptic weights are used than the ones described above.
Normalizing excitatory-to-excitatory (\verb|E→E|) connections to 1, Roxin et al's synaptic weights are:
\begin{verbatim}
    E→E: 1
    E→I: 18 (instead of 1)
    I→E: 36 (instead of 4)
    I→I: 31 (instead of 4)
\end{verbatim}
We will call these weights `Roxin2011'. The naive, `balancing' weights used in \cref{fig:1144-weights}, we will call `1144'.

Simulating a network with the `Roxin2011' weights indeed increases the firing rates of inhibitory neurons with respect to those of excitatory neurons (\cref{fig:roxin-weights}). The two firing rate distributions are however both as symmetrical as before.\footnote{
    When collating the two firing rate distributions from both neuron types, the resulting single distribution \emph{is} heavy-tailed, and even vaguely looks log-normal (but it is not). In \cite{Roxin2011DistributionFiringRates}, similar-looking, collated firing rate distributions are shown, and are called lognormal (but this is never quantified in the paper).
}

This imbalance between \verb|E→E| and \verb|E→I| synapses is not that rare: another modelling paper that also investigated EI-balance, \cite{Sadeh2021ExcitatoryinhibitoryBalanceModulates}, has the following synaptic weights. If our original weights are $[1, 1, 4, 4]$, then Sadeh and Clopath's weights are $[1, k, k, k]$, with $k = 4$ (in their "strong E-I coupling regime").

In the results that follow, we use both `1144' and the `Roxin2011' synaptic weights.

Finally, we also looked at drawing synaptic weights from a distribution (specifically, a lognormal one) instead of setting all weights of one type to be equal. This did not change the symmetry of the resulting firing rate distributions, so we did not pursue this further.


\section{Subsampling}

We simulate all 1000 neurons' voltages, but, to save memory and disk space,  do not record all these traces.\footnote{
    For a 10-minute simulation with a timestep of 0.1 ms, one voltage trace takes 48 MB (at 64 bit per sample). Our 1000 neurons thus take 48 GB -- and that is just for one simulation (one set of parameters).\newline
    We \emph{do} record the spike trains of all neurons. Saving just spike times takes considerably less space: a neuron spiking at 10 Hz for 10 minutes emits 6000 spikes, which, at 64 bit per timestamp, takes just  48 kB.\newline
    In other words,  at a 0.1 ms sample rate, a spike train occupies about 1000× less memory than the corresponding voltage trace.
}
In most experiments here, we recorded the voltage traces of 40 excitatory and 10 inhibitory neurons (5\% of all neurons).

Additionally, when performing connection tests on the inputs of a recorded neuron, we do not test the spiketrains of all 999 other neurons. Instead, we test only a (biased) sample of the possible inputs, to save processing time.
This sample is constructed as follows. We test all the a-priori known true direct inputs -- both excitatory and inhibitory -- and add a random sample of 40 not-directly-connected neurons.

On average, each neuron has ± 32 excitatory inputs and ± 8 inhibitory inputs. This means that, from the 1000 × 1000 possible connections, we only test about 4000, or 0.4\%.\footnote{
    Calculation behind these numbers:\\
    - 1000 neurons × 80\% excitatory × 4\% probability of an input connection = 32 excitatory inputs on average.\\
    - 50 `post' neurons (40 excitatory and 10 inhibitory voltage-recorded neurons) × 80 `pre' neurons (±40 connected + 40 unconnected) = 4000 tested connections.\\
    - To be precise, instead of 1000 × 1000, there are rather  $1000^2 - 1000$ possible connections, as we would not test for autapses.
}

\begin{figure}
    \subfloat{
        \includegraphics[w=0.74]{2022-09-01__1144_weights_23_0.png}
        \includegraphics[w=0.74]{2022-09-01__1144_weights_24_0.png}
    }
    \captionn[. ]
    {Performance of the STA `peak-to-peak' connection test in the random network}
    {10 minute recording, with `1144' weights. Each dot is the true positive rate for one `post' neuron. The gray dotted line indicates the p-value threshold α of 0.05.\\
    Source: \nburl{2022-09-01__1144_weights}.}
    \label{fig:net-perf}
    \vspace*{2em}
\end{figure}



\section{Connection testing}

We tested connections using the STA height (or `peak-to-peak') test. In the chronology of the PhD, the more advanced methods presented in the previous chapter were developed only after the network tests in this chapter were done. Applying the newer methods to the full network (and not just the N-to-1 setup) is a topic for further work.
% also: detrates at fixed threshold, instead of AUC.

\Cref{fig:net-perf} shows the performance of using the STA height as connection test in the network. The detection rates (one for each tested `post' neuron and `pre' neuron type) are shown at a fixed detection threshold $α$ of $0.05$. That is, a connection was classified as real if its STA was larger than 95 of its 100 `control' STAs, which where made by randomly shuffling the presynaptic spiketrains.\footnote{
    A spiketrain is shuffled by taking its inter-spike-intervals (ISIs), randomly shuffling those, and reconstructing a new spiketrain from the resulting shuffled ISIs.
}

Detection rates are broken down per type of both the presynaptic and postsynaptic neuron (excitatory or inhibitory). We find that inhibitory inputs are significantly easier to detect (independent of the type of the postsynaptic neuron). This is due to the synaptic weight of inhibitory inputs -- and thus their PSPs -- being $4×$ stronger than those of excitatory inputs.

The most interesting category in \cref{fig:net-perf} are the `Unconnected' detection rates (i.e. the false positive rates, FPRs), which are almost all higher than the detection threshold $α$. In theory, the FPR should equal the detection threshold α; and indeed this is what we roughly found in the N-to-1 experiments with STA-height shuffle test.\footnote{
    This is true almost by definition of the shuffle test: an unconnected spiketrain in the N-to-1 experiment has randomly generated spiketimes. Shuffling this random spiketrain creates more random spiketrains. Looking at the distribution of STA heights of these random spiketrains, the chance that the `real' STA (of the unconnected spiketrain) is in the top-α fraction of STA heights -- and that thus, it would be wrongly classified as a true input, i.e. a false positive -- is exactly α.
}
In the network, we thus detect unconnected inputs -- or to be more precise: not-directly-connected inputs -- at a rate higher than chance. This deserves more scrutiny, which we do in the next section.



\section{False positive detections}

As a summary of the above: when we test unconnected spiketrains as inputs to some neuron in the network, we falsely classify them as connected at a higher rate than would be expected if these spiketrains were fully random. So, they have a stronger-than-random STA. Or in other words, their spikes seem to have some influence on the target neuron's voltage, even though they are not direct inputs to it. In this section, we examine what is special about these false positive input neurons.

First, we found that all false positives (i.e. all not-directly-connected inputs that were nonetheless classified as connected) had a shortest path to the target neuron that consisted of only one in-between neuron. However, as we saw in \cref{fig:shortest-path-all}, this is not that special in this highly-interconnected random network.

But maybe there are \emph{more} such length-2 paths between the input neuron and the target neuron? We indeed found that this was the case when examining one target neuron and all its tested unconnected inputs. In \cref{fig:net-FP} (left), the blue distribution (number of length-2 paths to the target neuron, for false positive inputs) is higher than the orange distribution (same, but for true negatives).

Next, we looked at the type of in-between neurons on the shortest path. We found that for the false positive inputs, the in-between neuron was more likely to be inhibory (\cref{fig:net-FP}, right). This makes sense, as inhibitory inputs are stronger (they are also detected at a higher rate, as seen in \cref{fig:net-perf}).

\begin{figure}
    \subfloat{
        \includegraphics[w=0.74]{net-FP-length-2-paths}
        \includegraphics[w=0.74]{net-FP-inh}
    }
    \captionn
        {Characterizing what is different about false positive inputs}
        {All tested unconnected inputs to a selected target neuron in the network, with inputs correctly classfied as not-connected in orange, and the false positives in blue. Source: \nburl{2022-07-14__Unconnected-but-detected}.}
    \label{fig:net-FP}
    % \vspace*{2em} % to space from plot underneath.
\end{figure}


Finally, we looked at the firing rate of the in-between neurons (\cref{fig:net-FP-fr}). For false positives inputs, the in-between neurons were more active than for true negatives.


\begin{figure}
    \begin{sidecaption}
        {\textbf{False positives synapse onto high-firing direct inputs}.\\
        See \cref{fig:net-FP}.}
        [fig:net-FP-fr]
        \linelabel{fig:net-FP-fr}
        \includegraphics[w=0.74]{net-FP-fr}
    \end{sidecaption}
\end{figure}


% We tested spiketrain correlations as well: https://tfiers.github.io/phd/nb/2022-08-05__Spiketrain-correlations.html
% (but no signif difference, at any binsize)


In summary, we have some evidence (though it should be tested more thoroughly) that our higher-than-expected false positive rate is due to indirect inputs: neurons that synapse onto direct inputs. False positives synapse onto more such direct inputs, and those direct inputs have a stronger influence on the target neuron: a higher firing rate, and more likely to be inhibitory.

Further work could adapt our network topology to be more sparse, so that indirect connections are much rarer, and their effect easier to study.




\section{Conclusion}

In this chapter, we extended the N-to-1 testing setup to a full network of simulated neurons. We found that the STA height method could still detect connections. However, unlike in the N-to-1 setup -- with its fully independent inputs -- in the network, the false positive rate exceeded the detection threshold α. We hypothesized that this is due to indirect connections being detected as direct inputs, and found some preliminary evidence that that is indeed the case.

\marginpar{
    \includegraphics[w=1]{avg_STA}
    \captionof{figure}{
        Average STA of all true excitatory connections to recorded neurons in the network. Source: \nburl{2022-09-09__Conntest_with_template_matching}.
    }
    \label{fig:avg_STA}
}

One of the avenues pursued during the investigation of the indirect connections, was to plot their STA and compare it to the `ideal' STA\footnote{
    The hypothesis was that we would see a voltage bump twice as late as for direct connections, i.e. indicating a disynaptic connection. This investigation was inconclusive, but it did give us ideas for new connection detection methods.
}. This ideal STA (\cref{fig:avg_STA}) was made by using the ground truth connectivity and averaging all the STAs of the true direct connections. This clean STA gave us the idea to (1) model this shape, and use this model as some sort of `prior' for a stronger connection test; and (2) use it as a template to correlate STAs with. This led us to the new connection tests discussed in the next chapter.




\chapter{Network model}

\begin{itemize}
    \item Network connectivity, E/I balance, raster plots
    \item Too many possible connections to test them all → Subsampling
    \item Performance of last chapter's methods
    \item If time: experiment with a network that is less densely connected than our current fully-random one. Why? To better examine the effect of indirect connections / colliders (For the current connectivity, there are too many of those. But in a more realistic, 'localized' network, there are less, and so it seems easier to isolate and examine their effect).
\end{itemize}


\chapter{Discussion}

\section{Summary \& conclusions}

..

\section{Future work}

\begin{itemize}
    \item Test on real data
    \item Direct comparison with spikes-only methods
    \item More complexity in the testing setup: different transmission delays and time constants per synapse / neuron, plus:
    \item Short term synaptic plasticity. Bursting. Oscillations.
    \item Simulate different brain areas (different cell types and connectivity patterns). Simulate the same area, but in different states (up vs down, e.g.)
    \item New connection test method to try: something deep learning-based (we have infinite training data, given our simulation)
\end{itemize}

\References

\end{document}
